% Options for packages loaded elsewhere
\PassOptionsToPackage{unicode}{hyperref}
\PassOptionsToPackage{hyphens}{url}
\PassOptionsToPackage{dvipsnames,svgnames,x11names}{xcolor}
%
\documentclass[
]{agujournal2019}

\usepackage{amsmath,amssymb}
\usepackage{lmodern}
\usepackage{iftex}
\ifPDFTeX
  \usepackage[T1]{fontenc}
  \usepackage[utf8]{inputenc}
  \usepackage{textcomp} % provide euro and other symbols
\else % if luatex or xetex
  \usepackage{unicode-math}
  \defaultfontfeatures{Scale=MatchLowercase}
  \defaultfontfeatures[\rmfamily]{Ligatures=TeX,Scale=1}
\fi
% Use upquote if available, for straight quotes in verbatim environments
\IfFileExists{upquote.sty}{\usepackage{upquote}}{}
\IfFileExists{microtype.sty}{% use microtype if available
  \usepackage[]{microtype}
  \UseMicrotypeSet[protrusion]{basicmath} % disable protrusion for tt fonts
}{}
\makeatletter
\@ifundefined{KOMAClassName}{% if non-KOMA class
  \IfFileExists{parskip.sty}{%
    \usepackage{parskip}
  }{% else
    \setlength{\parindent}{0pt}
    \setlength{\parskip}{6pt plus 2pt minus 1pt}}
}{% if KOMA class
  \KOMAoptions{parskip=half}}
\makeatother
\usepackage{xcolor}
\setlength{\emergencystretch}{3em} % prevent overfull lines
\setcounter{secnumdepth}{5}
% Make \paragraph and \subparagraph free-standing
\ifx\paragraph\undefined\else
  \let\oldparagraph\paragraph
  \renewcommand{\paragraph}[1]{\oldparagraph{#1}\mbox{}}
\fi
\ifx\subparagraph\undefined\else
  \let\oldsubparagraph\subparagraph
  \renewcommand{\subparagraph}[1]{\oldsubparagraph{#1}\mbox{}}
\fi


\providecommand{\tightlist}{%
  \setlength{\itemsep}{0pt}\setlength{\parskip}{0pt}}\usepackage{longtable,booktabs,array}
\usepackage{calc} % for calculating minipage widths
% Correct order of tables after \paragraph or \subparagraph
\usepackage{etoolbox}
\makeatletter
\patchcmd\longtable{\par}{\if@noskipsec\mbox{}\fi\par}{}{}
\makeatother
% Allow footnotes in longtable head/foot
\IfFileExists{footnotehyper.sty}{\usepackage{footnotehyper}}{\usepackage{footnote}}
\makesavenoteenv{longtable}
\usepackage{graphicx}
\makeatletter
\def\maxwidth{\ifdim\Gin@nat@width>\linewidth\linewidth\else\Gin@nat@width\fi}
\def\maxheight{\ifdim\Gin@nat@height>\textheight\textheight\else\Gin@nat@height\fi}
\makeatother
% Scale images if necessary, so that they will not overflow the page
% margins by default, and it is still possible to overwrite the defaults
% using explicit options in \includegraphics[width, height, ...]{}
\setkeys{Gin}{width=\maxwidth,height=\maxheight,keepaspectratio}
% Set default figure placement to htbp
\makeatletter
\def\fps@figure{htbp}
\makeatother
\newlength{\cslhangindent}
\setlength{\cslhangindent}{1.5em}
\newlength{\csllabelwidth}
\setlength{\csllabelwidth}{3em}
\newlength{\cslentryspacingunit} % times entry-spacing
\setlength{\cslentryspacingunit}{\parskip}
\newenvironment{CSLReferences}[2] % #1 hanging-ident, #2 entry spacing
 {% don't indent paragraphs
  \setlength{\parindent}{0pt}
  % turn on hanging indent if param 1 is 1
  \ifodd #1
  \let\oldpar\par
  \def\par{\hangindent=\cslhangindent\oldpar}
  \fi
  % set entry spacing
  \setlength{\parskip}{#2\cslentryspacingunit}
 }%
 {}
\usepackage{calc}
\newcommand{\CSLBlock}[1]{#1\hfill\break}
\newcommand{\CSLLeftMargin}[1]{\parbox[t]{\csllabelwidth}{#1}}
\newcommand{\CSLRightInline}[1]{\parbox[t]{\linewidth - \csllabelwidth}{#1}\break}
\newcommand{\CSLIndent}[1]{\hspace{\cslhangindent}#1}

\usepackage{url} %this package should fix any errors with URLs in refs.
\usepackage{lineno}
\usepackage[inline]{trackchanges} %for better track changes. finalnew option will compile document with changes incorporated.
\usepackage{soul}
\linenumbers
\makeatletter
\makeatother
\makeatletter
\makeatother
\makeatletter
\@ifpackageloaded{caption}{}{\usepackage{caption}}
\AtBeginDocument{%
\ifdefined\contentsname
  \renewcommand*\contentsname{Table of contents}
\else
  \newcommand\contentsname{Table of contents}
\fi
\ifdefined\listfigurename
  \renewcommand*\listfigurename{List of Figures}
\else
  \newcommand\listfigurename{List of Figures}
\fi
\ifdefined\listtablename
  \renewcommand*\listtablename{List of Tables}
\else
  \newcommand\listtablename{List of Tables}
\fi
\ifdefined\figurename
  \renewcommand*\figurename{Figure}
\else
  \newcommand\figurename{Figure}
\fi
\ifdefined\tablename
  \renewcommand*\tablename{Table}
\else
  \newcommand\tablename{Table}
\fi
}
\@ifpackageloaded{float}{}{\usepackage{float}}
\floatstyle{ruled}
\@ifundefined{c@chapter}{\newfloat{codelisting}{h}{lop}}{\newfloat{codelisting}{h}{lop}[chapter]}
\floatname{codelisting}{Listing}
\newcommand*\listoflistings{\listof{codelisting}{List of Listings}}
\makeatother
\makeatletter
\@ifpackageloaded{caption}{}{\usepackage{caption}}
\@ifpackageloaded{subcaption}{}{\usepackage{subcaption}}
\makeatother
\makeatletter
\@ifpackageloaded{tcolorbox}{}{\usepackage[many]{tcolorbox}}
\makeatother
\makeatletter
\@ifundefined{shadecolor}{\definecolor{shadecolor}{rgb}{.97, .97, .97}}
\makeatother
\makeatletter
\makeatother
\ifLuaTeX
  \usepackage{selnolig}  % disable illegal ligatures
\fi
\IfFileExists{bookmark.sty}{\usepackage{bookmark}}{\usepackage{hyperref}}
\IfFileExists{xurl.sty}{\usepackage{xurl}}{} % add URL line breaks if available
\urlstyle{same} % disable monospaced font for URLs
\hypersetup{
  pdftitle={Preparing your manuscript},
  pdfauthor={Florian Börgel; Sven Karsten},
  colorlinks=true,
  linkcolor={blue},
  filecolor={Maroon},
  citecolor={Blue},
  urlcolor={Blue},
  pdfcreator={LaTeX via pandoc}}

\journalname{Geophysical Research Letters}

\draftfalse

\begin{document}
\title{Preparing your manuscript}

\authors{Florian Börgel\affil{1}, Sven Karsten\affil{1}}
\affiliation{1}{Leibniz-Institute for Baltic Sea Research Warnemünde, }
\correspondingauthor{Florian Börgel}{florian.boergel@io-warnemuende.de}


\begin{abstract}
The abstract (1) states the nature of the investigation and (2)
summarizes the important conclusions. The abstract should be suitable
for indexing. Your abstract should:

\begin{itemize}
\tightlist
\item
  Be set as a single paragraph.
\item
  Be less than 250 words for all journals except GRL, for which the
  limit is 150 words.
\item
  Not include table or figure mentions.
\item
  Avoid reference citations unless dependent on or directly related to
  another paper (e.g., companion, comment, reply, or commentary on
  another paper(s)). AGU's Style Guide discusses formatting citations in
  abstracts.
\item
  Define all abbreviations.
\end{itemize}
\end{abstract}

\section*{Plain Language Summary}
A Plain Language Summary (PLS) can be an incredibly effective science
communication tool. By summarizing your paper in non-technical terms,
you can explain your research and its relevance to a much broader
audience. A PLS is required for submissions to AGU Advances, G-Cubed,
GeoHealth, GRL, JAMES, JGR: Biogeosciences, JGR: Oceans, JGR: Planets,
JGR: Solid Earth, JGR: Atmospheres, Space Weather, and Reviews of
Geophysics, but optional for other journals. A PLS should be no longer
than 200 words and should be free of jargon, acronyms, equations, and
any technical information that would be unknown to people from outside
your scientific discipline. Read our tips for creating an effective PLS.


\ifdefined\Shaded\renewenvironment{Shaded}{\begin{tcolorbox}[breakable, sharp corners, borderline west={3pt}{0pt}{shadecolor}, frame hidden, boxrule=0pt, interior hidden, enhanced]}{\end{tcolorbox}}\fi

\hypertarget{introduction}{%
\section{Introduction}\label{introduction}}

River runoff is an important component of the global water cycle as it
comprises about one third of the precipitation over land areas
(Hagemann, 2020). Moreover, accurate runoff forecasting, especially over
extended periods, is pivotal for effective water resources management,
as highlighted by studies such as Yang et al.~(2018), Tan et al.~(2018),
and Fang et al.~(2019).

Over the past decades, models for long-term runoff forecasting have been
primarily bifurcated into physically based models and data-based models.
While the former attempts to emulate intricate and nonlinear physical
hydrological processes, the latter hinges on establishing statistical
models that delineate the relationship between large-scale climate
patterns and catchment runoff.

Machine Learning (ML) models, such as those employing artificial neural
networks, support vector machines, adaptive neuro-fuzzy inference
systems, and notably, Long Short-Term Memory (LSTM) neural networks,
have gained traction for long-term hydrological forecasting due to their
commendable performance (Humphrey et al 2016, Huang et al 2014, Ashrafi
et al 2017, Yuan et al 2018, Xu et al 2021).

LSTM networks, an evolution of the classical Recurrent Neural Networks
(RNNs), have shown stability and efficacy in sequence-to-sequence
predictions, such as using climatic indices for rainfall estimation or
long-term hydrological forecasting. However, a limitation of LSTMs is
their inability to effectively capture two-dimensional structures, an
area where Convolutional Neural Networks (CNNs) excel. Recognizing this,
we introduce the ConvLSTM, which integrates the strengths of both LSTM
and CNN, to extract spatiotemporal features from precipitation fields
for predicting river runoff in the Baltic Sea catchment, summarized by
97 inidivual rivers.

Modeling the Baltic Sea is to a large part the result of the quality of
the freshwater input, that is used for the simulation. Meier and
Kauker(2003) showed that decadal salinity variations of about 1 \(g\)
\(kg^{-1}\) are caused, inter alia, by annual runoff variations.
Further, Meier and Kauker (2003) showed that about 50 \% of the decadal
salinity variability can be explained by variations in freshwater input
into the Baltic Sea.

This paper delves into the application of deep learning, particularly
ConvLSTM, to the challenging task of precipitation nowcasting, a domain
yet to fully harness the potential of advanced machine learning
techniques. We present ConvLSTM as a novel solution to this
spatiotemporal sequence forecasting challenge, highlighting its
advantages and potential future applications.

\hypertarget{methods}{%
\section{Methods}\label{methods}}

\hypertarget{lstm-network}{%
\subsection{LSTM network}\label{lstm-network}}

The Long Short-Term Memory (LSTM), a specialized form of Recurrent
Neural Networks (RNNs), is specifically tailored for modeling temporal
sequences. Its unique design allows it to adeptly handle long-range
dependencies, setting it apart from traditional RNNs in terms of
accuracy. This prowess in modeling long-range dependencies has been
validated in various studies {[}12, 11, 17, 23{]}. The cornerstone of
LSTM's innovation is its memory cell, \(c_t\)t, which functions as a
repository of state information, also refered to as long-term memory.
This cell is accessed, modified, and reset through several
self-parameterized gates. As new inputs arrive, their information is
accumulated into the cell if the input gate is activated. The forget
gate \(f_t\) defines the percentage of the previous cell status that is
stored \(c_{t-1}\)\hspace{0pt}. The decision to propagate the latest
cell output, \(c_t\), to the final state, \(h_t\), is governed by the
output gate, \(o_t\). A significant advantage of this architecture is
the memory cell's ability to retain gradients. This mechanism addresses
the vanishing gradient problem, where, as input sequences elongate, the
influence of initial stages becomes harder to capture, causing gradients
of early input points to approach zero. The LSTM's activation function,
inherently recurrent, mirrors the identity function with a consistent
derivative of 1.0, ensuring the gradient remains stable throughout
backpropagation.

One LSTM cell hence maybe expressed as:

\[
\begin{aligned}
i_t &= \sigma(W_{xi} x_t + W_{hi} h_{t-1} + W_{ci} \circ c_{t-1} + b_i) \\
f_t &= \sigma(W_{xf} x_t + W_{hf} h_{t-1} + W_{cf} \circ c_{t-1} + b_f) \\
c_t &= f_t \circ c_{t-1} + i_t \circ \tanh(W_{xc} x_t + W_{hc} h_{t-1} + b_c) \\
o_t &= \sigma(W_{xo} x_t + W_{ho} h_{t-1} + W_{co} \circ c_t + b_o) \\
h_t &= o_t \circ \tanh(c_t)
\end{aligned}
\]

\hypertarget{convlstm-network}{%
\subsection{ConvLSTM network}\label{convlstm-network}}

The FC-LSTM, while adept in many scenarios, falters in encoding spatial
information when handling spatiotemporal data due to its reliance on
full connections in both input-to-state and state-to-state transitions.
Addressing this limitation, our design ensures that all inputs
\(X_1, \ldots, X_t\), cell outputs \(C_1, \ldots, C_t\), hidden states
\(H_1, \ldots, H_t\), and gates \(i_t, f_t, o_t\) of the ConvLSTM are 3D
tensors. The last two dimensions of these tensors represent spatial
dimensions, specifically rows and columns. Conceptually, these inputs
and states can be visualized as vectors positioned on a spatial grid.

In the ConvLSTM, the future state of a specific cell on this grid is
determined by the inputs and past states of its neighboring cells. This
spatial consideration is integrated by employing a convolution operator
in both state-to-state and input-to-state transitions, as illustrated in
Fig. 2. The foundational equations for ConvLSTM are:

\[
\begin{aligned}
i_t &= \sigma(W_{xi} \ast X_t + W_{hi} \ast H_{t-1} + W_{ci} \circ C_{t-1} + b_i) \\
f_t &= \sigma(W_{xf} \ast X_t + W_{hf} \ast H_{t-1} + W_{cf} \circ C_{t-1} + b_f) \\
C_t &= f_t \circ C_{t-1} + i_t \circ \tanh(W_{xc} \ast X_t + W_{hc} \ast H_{t-1} + b_c) \\
o_t &= \sigma(W_{xo} \ast X_t + W_{ho} \ast H_{t-1} + W_{co} \circ C_t + b_o) \\
H_t &= o_t \circ \tanh(C_t)
\end{aligned}
\]

Considering the states as representations of moving entities, a ConvLSTM
with a larger transitional kernel can capture rapid movements, while a
smaller kernel is apt for slower motions. Drawing parallels with
{[}16{]}, the traditional FC-LSTM's inputs, cell outputs, and hidden
states can also be envisioned as 3D tensors, with the last two
dimensions being unitary. Thus, the FC-LSTM can be seen as a specific
instance of ConvLSTM where all features are concentrated on a singular
cell.

To maintain consistency in the dimensions of states and inputs, padding
is essential before convolution operations. This padding, especially at
boundary points, can be interpreted as leveraging the state of external
surroundings for computations. Typically, all LSTM states are
initialized to zero before the first input, symbolizing a lack of
knowledge about future events. Similarly, zero-padding (as adopted in
this study) on hidden states sets the external world's state to zero,
implying no prior information about external conditions. This padding
approach allows for differential treatment of boundary points, proving
advantageous in many scenarios. For instance, in a system with a ball
moving and bouncing off unseen walls, the existence of these walls can
be inferred by observing the ball's repeated bounces, a deduction
challenging if boundary points share the same state transition dynamics
as inner points.

\hypertarget{acknowledgments}{%
\section{Acknowledgments}\label{acknowledgments}}

Phasellus interdum tincidunt ex, a euismod massa pulvinar at. Ut
fringilla ut nisi nec volutpat. Morbi imperdiet congue tincidunt.
Vivamus eget rutrum purus. Etiam et pretium justo. Donec et egestas sem.
Donec molestie ex sit amet viverra egestas. Nullam justo nulla,
fringilla at iaculis in, posuere non mauris. Ut eget imperdiet elit.

\hypertarget{open-research}{%
\section{Open research}\label{open-research}}

Phasellus interdum tincidunt ex, a euismod massa pulvinar at. Ut
fringilla ut nisi nec volutpat. Morbi imperdiet congue tincidunt.
Vivamus eget rutrum purus. Etiam et pretium justo. Donec et egestas sem.
Donec molestie ex sit amet viverra egestas. Nullam justo nulla,
fringilla at iaculis in, posuere non mauris. Ut eget imperdiet elit.

\hypertarget{references}{%
\section*{References}\label{references}}
\addcontentsline{toc}{section}{References}

\hypertarget{refs}{}
\begin{CSLReferences}{0}{0}
\vspace{1em}

\end{CSLReferences}



\end{document}
