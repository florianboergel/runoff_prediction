\documentclass[10pt,a4paper]{article}
\usepackage[utf8]{inputenc}
\usepackage[english]{babel}
\usepackage{amsmath}
\usepackage{amsfonts}
\usepackage{amssymb}
\usepackage{graphicx}
\usepackage{fourier}
\usepackage{amsbsy}
\usepackage[left=2cm,right=2cm,top=2cm,bottom=2cm]{geometry}
\usepackage[most]{tcolorbox}
\tcbset{colback=yellow!10!white, colframe=red!50!black, 
        highlight math style= {enhanced, %<-- needed for the ’remember’ options
            colframe=red,colback=red!10!white,boxsep=0pt}
        }


\author{Sven Karsten}

\newcommand{\del}[2]{\frac{\partial #1}{\partial #2}}

\begin{document}

\section{Runoff prediction architecture}

\subsection{ConvLSTM}

Input quantity $ X^t_k[x, y, \tau]$ consists of a $N_\tau$ long sequence (starting at $t-N_\tau+1$ and ending at time $t$)  of $N_k$ 2D maps with $N_x \times N_y$ pixels, i.e.
\begin{align}
x \in [1,N_x] \\
y \in [1,N_y] \\
\tau \in [1,N_\tau] \\
k \in [1,N_k]
\end{align}

For any gate $g \in [i, f, o, c] $ define the gate input as a spatial convolution of the instantaneous input with a learnable "Markovian" weighting kernel plus a "non-Markovian" part that consists of the hidden state from the previous time instance convoluted with another learnable kernel  
\begin{align}
g^t_h[x,y,\tau] & =  \mathcal{M}^{g}_{hk} [\xi,\eta]\, X^t_k[x-\xi, y-\eta, \tau] \\
& +  \mathcal{N}^{g}_{hh'}[\xi,\eta] \, H^t_{h'}[x-\xi, y-\eta, \tau-1] + \mathcal{B}^g_{h}[x,y]
\end{align}
%
The auxiliary dimensions are
\begin{align}
h,h' \in [1,N_h] \\
\xi\in [-(N_\xi-1)/2,(N_\xi-1)/2]\\
\eta\in [-(N_\eta-1)/2,(N_\eta-1)/2]
\end{align}
%
where the convolution kernel sizes $N_\xi, N_\eta$ are assumed to be odd.
%
Thus the adjustable objects have dimensions
\begin{align}
\mathcal{M}^{g}_{hk}[\xi,\eta] \in \mathbb{R}^{4\times(N_h\times N_c) \times(N_\xi \times N_\eta)} \\
\mathcal{N}^{g}_{hh'}[\xi,\eta] \in \mathbb{R}^{4\times(N_h\times N_h) \times (N_\xi \times N_\eta)} \\
\mathcal{B}^g_{h}[x,y] \in \mathbb{R}^{4 \times(N_h) \times (N_x\times N_y) }
\end{align}

%Concatenate Markovian and non-Markovian parts
%\begin{align}
%G^g_h[x,y,\tau,t]  = \left(\mathcal{M}^{g}_{hc} [\xi,\eta], \mathcal{N}^{g}_{hh'}[\xi,\eta]\right) \cdot \left(X_c[x-\xi, y-\eta, t-N_\tau+\tau], H_{h'}[x-\xi, y-\eta, \tau-1,t]\right)+ \mathcal{B}^g_{h}[x,y]
%\end{align}
%
We can now introduce compact matrix-vector and spatial convolution notation
\begin{align}
\vec{g}^t[\tau] = \left(\pmb{\mathcal{M}}^{g}, \pmb{\mathcal{N}}^{g}\right) \ast \left(\vec{X}^t[\tau], \vec{H}^t[ \tau-1]\right)+ \vec{\mathcal{B}}^g
\end{align}

The hidden state evolves according to the following iteration formula for LSTM networks
%
\begin{align}
I^t_h[x,y,\tau] & = \sigma (i^t_h[x,y,\tau]) \\
F^t_h[x,y,\tau] & = \sigma (f^t_h[x,y,\tau]) \\
O^t_h[x,y,\tau] & = \sigma (o^t_h[x,y,\tau]) \\
C^t_h[x,y,\tau] & = F^t_h[x,y,\tau] \, C^t_h[x,y,\tau-1] + I^t_h[x,y,\tau] \, \tanh (c^t_h[x,y,\tau]) \\
H^t_h[x,y,\tau] & = O^t_h[x,y,\tau]\, \tanh(C^t_h[x,y,\tau])
\end{align}
%
going over $\tau \in [1,N_\tau]$ with initial conditions $H_h[x,y,0,t]=0$ and $C_h[x,y,0,t]=0$.
%
The stated LSTM connections of the gates can be abbreviated by a functions $L$ that returns the hidden network state for the next iteration step
%
\begin{align}
\left(\vec{H}^t, \vec{C}^t \right) = L \left( \vec{X}^t[N_\tau], L \left( \vec{X}^t[N_\tau-1], \ldots L \left( \vec{X}^t[1], (\vec{H}^t[0], \vec{C}^t[0] \right) \ldots \right) \right) \ ,
\end{align}
where $L\left(\vec{X}^t[\tau], \left(\vec{H}^t[\tau-1], \vec{C}^t[\tau-1] \right) \right)$ represents LSTM network equations and the initial conditions are chosen as $\vec{H}^t[0]=\vec{C}^t[0]=0$ and with the abbreviation for final output $\vec{H}^t[N_\tau] \equiv \vec{H}^t$ and $\vec{C}^t[N_\tau] \equiv \vec{C}^t$

\subsection{Fully connected layer}

Choose $a \in [1, 512]$, $b \in [1, 256]$ then
%
\begin{align}
R_r[t] = \mathcal{W}^{3}_{rb}\mathrm{ReLU}\left(\mathcal{W}^{2}_{ba}\mathrm{ReLU}\left(\mathcal{W}^{1}_{ah} [x,y] H^t_h[x,y] + \mathcal{B}^1_a\right) + \mathcal{B}^2_b \right) + \mathcal{B}^3_r
\end{align}
%
with $r \in [1, 97]$.


\section{Interpretation of a linearized model}

We can write our final result $R_r[t]$ for the $r$-th river as a function $M_r$ (for model) which itself is a nested function call of the $r$-th component of the fully connected layer $F_r$ and the ConvLSTM $K$ on a given input time sequence  $\{X^t_k[x,y,\tau]\}$, i.e.
\begin{tcolorbox}[ams align]
R_r[t] = M_r(\{X^t_k[x,y,\tau]\}) = F_r(K(\{X^t_k[x,y,\tau]\}))
\end{tcolorbox}
%
The idea is to approximate each $R_r[t]$ like
%
\begin{tcolorbox}[ams align]
R_r[t] \approx A_r[t] = a_r + \omega_{rk}[x,y,\tau] X^t_k[x,y,\tau]
\end{tcolorbox}
%
or expressed in continuous variables
\begin{align}
A_r(t) = a_r + \intop_{0}^{\tau_{\text{max}}} \text{d}\tau \intop_{\mathbb{R}^2} \text{d}x \text{d}y \, \omega_{rk}(x,y,\tau) X^t_k(x,y,\tau)
\end{align}
%
which means that the runoff is approximately determined by the spatially dependent memory kernel that weights the influence of each time instance of each channel back to the time $t-\tau_{\text{max}}$.   

The offset $a_r$ and the spatiotemporal weighting $\omega_{rk}[x,y,\tau]$ should be given by the first order Taylor expansion around a reference input sequence $\{X_k[x,y,\tau]\}$ that is not connected to any specific time $t$ (thus no superscript $t$) i.e.
%
\begin{align}
A_r[t] = M_r(\{X_k[x,y,\tau]\}) + \del{ M_r(\{X_k[x,y,\tau]\})}{X_k[x,y,\tau]}(X^t_k[x,y,\tau]-X_k[x,y,\tau])
\end{align}
thus
\begin{tcolorbox}[ams align]
a_r = M_r(\{X_k[x,y,\tau]\})) - \del{ M_r(\{X_k[x,y,\tau]\})}{X_k[x,y,\tau]}X_k[x,y,\tau], \quad \omega_{rk}[x,y,\tau] = \del{ M_r(\{X_k[x,y,\tau]\})}{X_k[x,y,\tau]}
\end{tcolorbox}
%
The upcoming task is hence to calculate the derivative of the model network with respect to the input.

In order to keep the overview it is beneficial to write the output of each function in the nest as individual variables, i.e.
\begin{tcolorbox}[ams align]
\{H_h[x,y]\} = K(\{X_k[x,y,\tau]\}) \\
\phi_r = F_r(\{H_h[x,y]\})
\end{tcolorbox}

%
Note that the superscript $t$ is now omitted in the internal states since we calculate these quantities with respect to the reference sequence $\{X_k[x,y,\tau]\}$ that should not be connected to any specific time.

As a first step we can now further evaluate the derivative via the chain rule
\begin{align}
\omega_{rk}[x,y,\tau] = \del{ M_r(\{X_k[x,y,\tau]\})}{X_k[x,y,\tau]} = \del{\phi_r}{H_{h'}[x',y']} \del{H_{h'}[x',y']}{X_k[x,y,\tau]}
\end{align}
%
The next task is to unfold the dependence of $H_{h}[x,y]$ on $\{X_k[x,y,\tau]\}$.
%
For that purpose we define the action of one ConvLSTM cell as function $L$. 
Then $\epsilon_{sh}[x,y]$ is the $N_\tau$-fold nested function
\begin{align}
H_{h}[x,y] = K(\{X_k[x,y,\tau]\}) = L(\{X_k[x,y,N_\tau]\},L(\{X_k[x,y,N_\tau-1]\},\ldots, \{X_k[x,y,2]\}, L(\{X_k[x,y,1]\},0)\ldots ))
\end{align}
%
We can now define the output of the $L$-calls recursively for a fixed $\tau$ as 
%
\begin{align}
\{\kappa_{sh}[x,y,\tau]\} = L\left(\{X_k[x,y,\tau]\}, \{\kappa_{sh}[x,y,\tau-1]\} \right)
\end{align}
where the additional index $s$ labels the two outputs of the $L$-function, i.e. $C_{h}[x,y,\tau]=\kappa_{1h}[x,y,\tau]$ and $H_{h}[x,y,\tau]=\kappa_{2h}[x,y,\tau]$.
The initial condition is $\kappa_{sh}[x,y,0]=0$ and the endpoint is by construction $\kappa_{2h}[x,y,N_\tau]=H_{h}[x,y]$.
%
Now we can use the chain rule to calculate for a specific $\tau$
%
\begin{align}
 \del{H_{h'}[x',y']}{X_k[x,y,\tau]} = \left. \del{\kappa_{s' h'}[x',y',N_\tau]}{X_k[x,y,\tau]} \right |_{s'=2} = \left. \left( \prod_{j=\tau}^{N_\tau-1} \del{\kappa_{s_{j+1}h_{j+1}}[x_{j+1},y_{j+1},j+1]}{\kappa_{s_{j}h_{j}}[x_{j},y_{j},j]} \right) \right |_{s_{N_\tau} = s' = 2} \del{\kappa_{s_{\tau}h_{\tau}}[x_{\tau},y_{\tau},\tau]}{X_k[x,y,\tau]}
\end{align}
with the boundary conditions for the auxiliary indices $x_{N_\tau}\equiv x', y_{N_\tau} \equiv y', h_{N_\tau} \equiv h', s_{N_\tau} \equiv s'$.
%
This can be inserted into the formula for $\omega_{rk}[x,y,\tau]$ yielding
\begin{tcolorbox}[ams align]
\omega_{rk}[x,y,\tau] = \del{\phi_r}{H_{h'}[x',y']} \left.\left( \prod_{j=\tau}^{N_\tau-1} \del{\kappa_{s_{j+1}h_{j+1}}[x_{j+1},y_{j+1},j+1]}{\kappa_{s_{j}h_{j}}[x_{j},y_{j},j]} \right) \right |_{s_{N_\tau} = s' = 2} \del{\kappa_{s_{\tau}h_{\tau}}[x_{\tau},y_{\tau},\tau]}{X_k[x,y,\tau]}
\end{tcolorbox}

At this point we have to evaluate the appearing derivatives explicitly.
%
We start with the dependence of the $\kappa$ at sequence time $j$ on the previous $\kappa$ at $j-1$.
%
\begin{align}
\del{\kappa_{1h'}[x',y',j]}{\kappa_{1h}[x,y,j-1]}  =\del{C_{h'}[x',y',j]}{C_h[x,y,j-1]} & = F_{h'}[x',y',j]\delta_{hh'}\delta_{xx'}\delta_{yy'} \\
\del{\kappa_{2h'}[x',y',j]}{\kappa_{1h}[x,y,j-1]}  = \del{H_{h'}[x',y',j]}{C_h[x,y,j-1]} & = \frac{O_{h'}[x',y',j] F_{h'}[x',y',j]}{\cosh^2(C_{h'}[x',y',j])}  \delta_{hh'}\delta_{xx'}\delta_{yy'} \\
\del{\kappa_{1h'}[x',y',j]}{\kappa_{2h}[x,y,j-1]} = \del{C_{h'}[x',y',j]}{H_h[x,y,j-1]} & = 
\dot{\sigma}(f_{h'}[x',y',j])\del{f_{h'}[x',y',j]}{H_h[x,y,j-1]}C_{h'}[x',y',j-1]\\
& + \dot{\sigma}(i_{h'}[x',y',j])\del{i_{h'}[x',y',j]}{H_h[x,y,j-1]} \tanh(c_{h'}[x',y',j])\\
& +  \frac{\sigma(i_{h'}[x',y',j])}{\cosh^2(c_{h'}[x',y',j])} \del{c_{h'}[x',y',j]}{H_h[x,y,j-1]} \\
\del{\kappa_{2h'}[x',y',j]}{\kappa_{2h}[x,y,j-1]} = \del{H_{h'}[x',y',j]}{H_h[x,y,j-1]} & = 
\dot{\sigma}(o_{h'}[x',y',j])\del{{o_{h'}[x',y',j]}}{H_h[x,y,j-1]}\tanh(C_{h'}[x',y',j]) \\
& + \frac{\sigma(o_{h'}[x',y',j])}{\cosh^2(C_{h'}[x',y',j])} \del{C_{h'}[x',y',j]}{H_h[x,y,j-1]}
\end{align}
%
Note that $ \partial \sigma(x)/ \partial {x} = \sigma(x)(1-\sigma(x)) =: \dot{\sigma}(x)$ and $\partial \tanh(x)/ \partial {x} = 1/\cosh^2(x)$.
%
The remaining task is to calculate the derivative of the gate functions $g_h[x,y,j]$ with respect to the previous hidden state $H_h[x,y,j-1]$, i.e. 
%
\begin{align}
\del{{g_{h'}[x',y',j]}}{H_h[x,y,j-1]} = \mathcal{N}^{g}_{h'h}[x'-x,y'-y]
\end{align}
%
Then we finally have
\begin{tcolorbox}[ams align]
\del{\kappa_{1h'}[x',y',j]}{\kappa_{1h}[x,y,j-1]} & = F_{h'}[x',y',j]\delta_{hh'}\delta_{xx'}\delta_{yy'} \\
\del{\kappa_{2h'}[x',y',j]}{\kappa_{1h}[x,y,j-1]} & = \frac{O_{h'}[x',y',j] F_{h'}[x',y',j]}{\cosh^2(C_{h'}[x',y',j])}  \delta_{hh'}\delta_{xx'}\delta_{yy'} \\
\del{\kappa_{1h'}[x',y',j]}{\kappa_{2h}[x,y,j-1]} & = 
\dot{\sigma}(f_{h'}[x',y',j]) C_{h'}[x',y',j-1] \mathcal{N}^{f}_{h'h}[x'-x,y'-y]\\
& + \dot{\sigma}(i_{h'}[x',y',j]) \tanh(c_{h'}[x',y',j])\mathcal{N}^{i}_{h'h}[x'-x,y'-y]\\
& +  \frac{\sigma(i_{h'}[x',y',j])}{\cosh^2(c_{h'}[x',y',j])} \mathcal{N}^{c}_{h'h}[x'-x,y'-y] \\
\del{\kappa_{2h'}[x',y',j]}{\kappa_{2h}[x,y,j-1]} & = 
\dot{\sigma}(o_{h'}[x',y',j])\tanh(C_{h'}[x',y',j]) \mathcal{N}^{o}_{h'h}[x'-x,y'-y] \\
& + \frac{\sigma(o_{h'}[x',y',j])}{\cosh^2(C_{h'}[x',y',j])} \del{\kappa_{1h'}[x',y',j]}{\kappa_{2h}[x,y,j-1]}
\end{tcolorbox}
%

The next task is to calculate the dependence of the $\kappa$ on the input quantity $X$.
\begin{align}
\del{\kappa_{1h'}[x',y',\tau]}{X_k[x,y,\tau]} = \del{C_{h'}[x',y',\tau]}{X_k[x,y,\tau]} & = 
\dot{\sigma}(f_{h'}[x',y',\tau]) \del{f_{h'}[x',y',\tau]}{X_k[x,y,\tau]} C_{h'}[x',y',\tau-1] \\
& + \dot{\sigma}(i_{h'}[x',y',\tau]) \del{i_{h'}[x',y',\tau]}{X_k[x,y,\tau]} \tanh(c_{h'}[x',y',\tau])\\
& +  \frac{\sigma(i_{h'}[x',y',\tau])}{\cosh^2(c_{h'}[x',y',\tau])} \del{c_{h'}[x',y',\tau]}{X_k[x,y,\tau]} \\
\del{\kappa_{2h'}[x',y',\tau]}{X_k[x,y,\tau]}  = \del{H_{h'}[x',y',\tau]}{X_k[x,y,\tau]} &=
\dot{\sigma}(o_{h'}[x',y',\tau])\del{{o_{h'}[x',y',\tau]}}{X_k[x,y,\tau]}\tanh(C_{h'}[x',y',\tau]) \\
& + \frac{\sigma(o_{h'}[x',y',\tau])}{\cosh^2(C_{h'}[x',y',\tau])} \del{C_{h'}[x',y',\tau]}{X_k[x,y,\tau]}
\end{align} 
%
The remaining task is to calculate the derivative of the gate functions $g_h[x,y,\tau]$ with respect to the current input $X_k[x,y,\tau]$, i.e. 
%
\begin{align}
\del{{g_{h'}[x',y',\tau]}}{X_k[x,y,\tau]} = \mathcal{M}^{g}_{h'k}[x'-x,y'-y]
\end{align}
%
Inserted into the equations above yields
%
\begin{tcolorbox}[ams align]
\del{\kappa_{1h'}[x',y',\tau]}{X_k[x,y,\tau]}  & = 
\dot{\sigma}(f_{h'}[x',y',\tau])  C_{h'}[x',y',\tau-1] \mathcal{M}^{f}_{h'k}[x'-x,y'-y] \\
& + \dot{\sigma}(i_{h'}[x',y',\tau])  \tanh(c_{h'}[x',y',\tau])\mathcal{M}^{i}_{h'k}[x'-x,y'-y]\\
& +  \frac{\sigma(i_{h'}[x',y',\tau])}{\cosh^2(c_{h'}[x',y',\tau])} \mathcal{M}^{c}_{h'k}[x'-x,y'-y] \\
\del{\kappa_{2h'}[x',y',\tau]}{X_k[x,y,\tau]}  &=
\dot{\sigma}(o_{h'}[x',y',\tau])\tanh(C_{h'}[x',y',\tau])\mathcal{M}^{o}_{h'k}[x'-x,y'-y] \\
& + \frac{\sigma(o_{h'}[x',y',\tau])}{\cosh^2(C_{h'}[x',y',\tau])} \del{\kappa_{1h'}[x',y',\tau]}{X_k[x,y,\tau]} 
\end{tcolorbox} 
%

Now that we have expressions for the derivatives we can give them names
\begin{tcolorbox}[ams align]
\Lambda_{s's,h'h}[x', x, y', y, j] = \del{\kappa_{s'h'}[x',y',j]}{\kappa_{sh}[x,y,j-1]}  \\
\lambda_{s'h',k}[x', x, y', y, \tau] = \del{\kappa_{s'h'}[x',y',\tau]}{X_k[x,y,\tau]} 
\end{tcolorbox}
and insert it into the full expression
\begin{align}
\omega_{rk}[x,y,\tau] = \del{\phi_r}{\tilde{H}_{h'}[x',y']} \left. \left( \prod_{j=\tau}^{N_\tau-1} \Lambda_{s_{j+1}s_j,h_{j+1}h_j}[x_{j+1}, x_j, y_{j+1}, y_j, j+1] \right) \right |_{s_{N_\tau} = s' = 2} 
\lambda_{s_\tau h_\tau,k}[x_\tau, x, y_\tau, y, \tau]
\end{align}
%
To further shorten the expression we can formally carry out the product as 
\begin{align}
\chi_{s_{n} h_{n},k}[x_{n}, x, y_{n}, y, n, \tau] = \left( \prod_{j=\tau}^{n-1} \Lambda_{s_{j+1}s_j,h_{j+1}h_j}[x_{j+1}, x_j, y_{j+1}, y_j, j+1] \right)
\lambda_{s_\tau h_\tau,k}[x_\tau, x, y_\tau, y, \tau]
\end{align}
with $\tau \leq n \leq N_\tau$ then
%
\begin{tcolorbox}[ams align]
\chi_{s_{n+1} h_{n+1},k}[x_{n+1}, x, y_{n+1}, y, n+1, \tau] =  \Lambda_{s_{n+1}s_n,h_{n+1}h_n}[x_{n+1}, x_n, y_{n+1}, y_n, n+1] 
\chi_{s_{n} h_{n},k}[x_{n}, x, y_{n}, y, n, \tau]
\end{tcolorbox}
%
for $n=\tau,\ldots N_\tau-1$ and $\chi_{s_{\tau} h_{\tau},k}[x_{\tau}, x, y_{\tau}, y, \tau, \tau] = \lambda_{s_\tau h_\tau,k}[x_\tau, x, y_\tau, y, \tau]$.


%
and insert into the full expression
%
\begin{align}
\omega_{rk}[x,y,\tau] = \del{\phi_r}{H_{h'}[x',y']} \chi_{2,h',k}[x', x, y', y, N_\tau, \tau]
\end{align}
%
If we define the symbol 
%
\begin{tcolorbox}[ams align]
\Xi_{h'k}[x', x, y', y, \tau] =  \chi_{2,h',k}[x', x, y', y, N_\tau, \tau]
\end{tcolorbox}
%
we can write
\begin{tcolorbox}[ams align]
\omega_{rk}[x,y,\tau] = \del{\phi_r}{H_{h'}[x',y']} \Xi_{h'k}[x', x, y', y, \tau]
\end{tcolorbox}

The remaining task is to find the derivative of the fully connected layser with respect to the decoder's output.
%
For this purpose we define gain the output variables of the nested ReLU functions as individual variables, i.e.
\begin{tcolorbox}[ams align]
\rho^1_a & = \mathcal{W}^{1}_{ah} [x,y] H_h[x,y] + \mathcal{B}^1_a \\
\rho^2_b & = \mathcal{W}^{2}_{ba} \mathrm{ReLU}\left(\rho^1_a\right) + \mathcal{B}^2_b \\
\phi_r & = \mathcal{W}^{3}_{rb} \mathrm{ReLU}\left(\rho^2_b \right) + \mathcal{B}^3_r
\end{tcolorbox}
%
With these definitions we can write after using the chain rule
\begin{align}
\del{\phi_r}{H_{h'}[x',y']} = \del{\phi_r}{\rho^2_{b'}} \del{\rho^2_{b'}}{\rho^1_{a'}} \del{\rho^1_{a'}}{\tilde{H}_{h''}[x'',y'']}
\end{align}
The individual derivatives can be easily carried out by noting $\partial \mathrm{ReLU}(x) / \partial x = \theta(x)$ which is the Heaviside step function
\begin{align}
\del{\phi_r}{\rho^2_{b'}} & = \mathcal{W}^{3}_{rb'} \theta \left( \rho^2_{b'} \right)  \\
 \del{\rho^2_{b'}}{\rho^1_{a'}} & =  \mathcal{W}^{2}_{b'a'} \theta \left( \rho^1_{a'} \right) \\
 \del{\rho^1_{a'}}{H_{h'}[x',y']} & = \mathcal{W}^{1}_{a'h'} [x',y']
\end{align}
%
\begin{tcolorbox}[ams align]
\del{\phi_r}{H_{h'}[x',y']} = \mathcal{W}^{3}_{rb'} \theta \left( \rho^2_{b'} \right) \cdot  \mathcal{W}^{2}_{b'a'} \theta \left( \rho^1_{a'} \right) \cdot \mathcal{W}^{1}_{a'h'} [x',y'] = \Omega_{r h'}[x',y']
\end{tcolorbox}

As the final result we can then write 
%
\begin{tcolorbox}[ams align]
\omega_{rk}[x,y,\tau] = \Omega_{r h'}[x',y'] \cdot \Xi_{h'k}[x', x, y', y, \tau]
\end{tcolorbox}

\subsection{Recipe}

Evaluating the result from the right to the left suggests the following recipe:

\begin{itemize}
\item take a specific reference sequence $\{X_k[x,y,\tau]\}$ and go over $\tau \in [1,N_\tau]$ in ascending order
\begin{itemize}
\item calculate for each $\tau$
$$ \chi_{s_\tau h_\tau,k}[x_\tau, x, y_\tau, y, \tau, \tau] = \lambda_{s_\tau h_\tau,k}[x_\tau, x, y_\tau, y, \tau] $$
where $\lambda$ is a function of the current gates $g_{h_\tau}[x_\tau,y_\tau,\tau]$, the last cell state $C_{h_\tau}[x_\tau,y_\tau,\tau-1]$ and the parameters $\mathcal{M}^{g}_{h_\tau k}[x_\tau-x,y_\tau-y]$, \begin{tcolorbox} expense: $N_k (2 N_h N_x^2 N_y^2)$ \end{tcolorbox}
\item go then over $n \in [\tau, N_\tau-1]$ in ascending order
\begin{itemize}
\item for each $n$ calculate the tensor $\Lambda_{s_{n+1}s_n,h_{n+1}h_n}[x_{n+1}, x_n, y_{n+1}, y_j, n+1]$ as a function of the input gates and hidden states of the $n+1$-th and $n$-th ConvLSTM cells and the parameters $\mathcal{M}^{g}_{h_\tau k}[x_\tau-x,y_\tau-y]$ and $\mathcal{N}^{g}_{h_\tau k}[x_\tau-x,y_\tau-y]$
and perform the summation over $s_n, h_n, x_n, y_n$ for each $k, s_{n+1}, h_{n+1}, x, x_{n+1}, y, y_{n+1}$
$$ \chi_{s_{n+1} h_{n+1},k}[x_{n+1}, x, y_{n+1}, y, n+1, \tau] =  \Lambda_{s_{n+1}s_n,h_{n+1}h_n}[x_{n+1}, x_n, y_{n+1}, y_n, n+1] 
\chi_{s_{n} h_{n},k}[x_{n}, x, y_{n}, y, n, \tau]$$
\begin{tcolorbox} expense: $N_k (2 N_h N_x^2 N_y^2) (2 N_h N_x N_y)$ \end{tcolorbox}
\end{itemize}
\item the final result is $\chi_{s_{N_\tau} h_{N_\tau},k}[x_{N_\tau}, x, y_{N_\tau}, y, N_\tau, \tau]$
\begin{tcolorbox} accumulated expense: $N_k (2 N_h N_x^2 N_y^2) +(N_\tau - \tau) N_k (2 N_h N_x^2 N_y^2) (2 N_h N_x N_y)$ \end{tcolorbox}
\item calculate $\tilde{\chi}_{s_{N_\tau}h_{N_\tau}, h''}[x'', x_{N_\tau}, y'', y_{N_\tau}]$ as function of the decoder gates $\tilde{g}_{h''}[x'',y'']$, the encoder's last cell state $C_{h_{N_\tau}}[x_{N_\tau},y_{N_\tau},{N_\tau}]$ and the parameters $\tilde{\mathcal{Q}}^{g}_{h'' k}[x''-x_{N_\tau},y''-y_{N_\tau}]$ $\to$ expense: $2 N_h^2 N_x^2 N_y^2$
\item accumulated expense: $N_k (2 N_h N_x^2 N_y^2) +(N_\tau - \tau) N_k (2 N_h N_x^2 N_y^2) (2 N_h N_x N_y)+2 N_h^2 N_x^2 N_y^2$
\item then perform the summation over $s_{N_\tau}, h_{N_\tau}, x_{N_\tau}, y_{N_\tau}$ for each $h'', k, x'', x, y'', y$
$$\Xi_{h''k}[x'', x, y'', y, \tau] = \tilde{\chi}_{s_{N_\tau}h_{N_\tau}, h''}[x'', x_{N_\tau}, y'', y_{N_\tau}] \chi_{s_{N_\tau} h_{N_\tau},k}[x_{N_\tau}, x, y_{N_\tau}, y, N_\tau, \tau] $$
\begin{tcolorbox} expense: $N_k (N_h N_x^2 N_y^2) (2 N_h N_x N_y)$ \end{tcolorbox}
\item next calculate $\Omega_{r h''}[x'',y'']$ as a function of the decoder's output $\tilde{H}_{h''}[x'',y'']$ and the parameters $\mathcal{W}^{1}_{a'h''}[x'', y''], \mathcal{W}^{2}_{b'a'}$ and $ \mathcal{W}^{3}_{rb'}$
\begin{tcolorbox} expense: $N_h N_x N_y$ \end{tcolorbox}
\item finally perform the summation over $h'', x'', y''$ for each $k,x,y$
$$ \omega_{rk}[x,y,\tau] = \Omega_{r h''}[x'',y''] \cdot \Xi_{h''k}[x'', x, y'', y, \tau] $$
\begin{tcolorbox} expense: $N_k (N_h N^2_x N^2_y)$ \end{tcolorbox}

\end{itemize}

\begin{tcolorbox}
accumulated expense: 
\begin{align}
 N_\tau & \left[ N_k (2 N_h N_x^2 N_y^2) +(N_\tau - \tau) N_k (2 N_h N_x^2 N_y^2) (2 N_h N_x N_y) \right. \nonumber \\ 
 & \left.  + 2 N_h^2 N_x^2 N_y^2 + N_k (N_h N_x^2 N_y^2) (2 N_h N_x N_y) + N_h N_x N_y + N_k (N_h N^2_x N^2_y) \right]
\end{align}
\end{tcolorbox}
\end{itemize}

\subsection{Choice of the reference sequence $\{X_k[x,y,\tau]\}$}

\subsubsection{Average over all sequences in the data set}

Suppose the number of all sequences with length $N_\tau$ in the data set is $N_s$
\begin{align}
\bar{A}_r[t] & = \frac{1}{N_s}\sum_{\{X_k[x,y,\tau]\} \in \{X^t_k[x,y,\tau]\}} \left( M_r(\{X_k[x,y,\tau]\}) + \del{ M_r(\{X_k[x,y,\tau]\})}{X_k[x,y,\tau]}(X^t_k[x,y,\tau]-X_k[x,y,\tau]) \right) \\
& = \left(  \frac{1}{N_s}\sum_{\{X_k[x,y,\tau]\} \in \{X^t_k[x,y,\tau]\}} M_r(\{X_k[x,y,\tau]\})) - \del{ M_r(\{X_k[x,y,\tau]\})}{X_k[x,y,\tau]}X_k[x,y,\tau] \right) \\
& + \left( \frac{1}{N_s} \sum_{\{X_k[x,y,\tau]\} \in \{X^t_k[x,y,\tau]\}} \del{ M_r(\{X_k[x,y,\tau]\})}{X_k[x,y,\tau]} \right) X^t_k[x,y,\tau] \\
& = \bar{a}_r + \bar{\omega}_{rk}[x,y,\tau] X^t_k[x,y,\tau] 
\end{align}

\subsubsection{Optimal sequence}

Find a specific sequence $\{X^*_k[x,y,\tau]\}$ such that
\begin{align}
{A}^*_r[t] = M_r(\{X^*_k[x,y,\tau]\}) + \del{ M_r(\{X^*_k[x,y,\tau]\})}{X^*_k[x,y,\tau]}(X^t_k[x,y,\tau]-X^*_k[x,y,\tau])
\end{align}
minimizes the integrated squared error
\begin{align}
\sum_t (R_r[t]-{A}^*_r[t])^2
\end{align}

\end{document}